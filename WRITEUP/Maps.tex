\documentclass[]{emulateapj}
\usepackage{amsmath}
\begin{document}
\author{Draft}


\section{Continuum}


\begin{figure*}[tbph]
\centering
\includegraphics[width=0.80\textwidth]{}
\caption{Left: Contours of the CARMA BLAH GHz continuum emission (red) in the foreground radio galaxy 3C220.3 overlaid on the image and contours of the VLA 6 GHz continuum emission (lime). The rms in the VLA 6 GHz continuum image is $\sigma$ = 0.0641 mJy beam$^{-1}$. The synthesized beam size is 0$\farcs$596 $\times$ 0$\farcs$229, at P.A. 75.79$\degr$. Lime contour levels start at $\pm$4$\times\sigma$ = x mJy beam$^{-1}$ and increment at steps of $\pm$2$\sigma$ for the VLA data; red contour levels start at $\pm$2$\sigma$, incrementing at steps of $\pm$1$\sigma$ = 0.24 mJy beam$^{-1}$.
Right:
Contour map of the 90.7 GHz continuum emission. The beam size is 16\farcs67$\pm
$0\farcs69 $\times$ 7\farcs17$\pm$1\farcs60, at P.A. = -57.9$\degr$. The contour levels start at $\pm$2$\sigma$ and increment at steps of $\pm$1$\sigma$ =0.5 mJy beam$^{-1}$.\label{fig:cont}}
\end{figure*}


\section{CO($J$ = 3 $\rightarrow$ 2) line emission}
%- ! Parameters and Errors:
%- ! MFIT%PAR[01]  
%- ! MFIT%PAR[02]  
%- ! MFIT%PAR[03]  
%- ! MFIT%PAR[04]  
%- channel [, ]
%- velocity range [,] km/s
%- Integrated line flux =
%- rms =

%We have detected unresolved CO($J$ = 3 $\rightarrow$ 2) line emission toward SMM J0939+8315 at peak flux
%of blah sigma  of $ \sigma$ = blah mJy beam$^{-1}$.
We detect unresolved CO($J$ = 3 $\rightarrow$ 2) line emission toward the background galaxy in RXJ1131-1231. 
We extract the spectrum and fit a single-component Gaussian as shown in Figure \ref{fig:mom0}, yielding peak flux density of BLAH $\pm$ BLAH mJy, the FWHM of the Gaussian fit is BLAH $\pm$ BLAH km s$^{-1}$. 
The spatial extent of the  is shown in the SMA 1 mm dust continuum in Figure \ref{fig:mom0}, where the synthesized beam size of the SMA observation is 1\farcs42$\times$1\farcs16, at P.A. -34.0\degr. Since the angular resolution of our CO measurement is 13$\farcs$57 $\pm$ 1\farcs4 $\times$ 5\farcs77 $\pm$ 3\farcs1, the CO($J$ = 3 $\rightarrow$ 2) detection in the lensed galaxy is therefore spatially unresolved. 
We construct the velocity-integrated (moment-0) map of the CO($J$ = 3 $\rightarrow$ 2) emission using the uv-continuum subtracted data, the velocity-integrated CO($J$ = 3 $\rightarrow$ 2) line flux is BLAH $\pm$ BLAH Jy km s$^{-1}$.

\begin{figure*}[tbph]
\centering
\includegraphics[width=0.8\textwidth]{../OverlayFig/CO32mom0}
\includegraphics[width=0.65\textwidth]{../OverlayFig/2.685bin2_arcNS_further_Sblob}
\caption{The central cross represents the coordinates for alignment between plots. The contour starts at $\pm$2$\sigma$, incrementing at
steps of $\pm$1$\sigma$. Top Left: Continuum-subtracted moment-0 map of the CO($J$ = 3 $\rightarrow$ 2) line emission in the background SMG with $\sigma$ = 0.7 Jy km s$^{-1}$ beam$^{-1}$. The angular resolution is 13$\farcs$57 $\times$ 5\farcs77, at P.A. = -60.67$\degr$.
Top Right: Contours of CO($J$ = 3 $\rightarrow$ 2) emission (red) overlaid on the SMA 1 mm dust continuum (lime contours) with $\sigma$=0.836951 mJy beam$^{-1}$. The beam size of the SMA data is 1\farcs42$ \times $1\farcs16, P.A. -34.0\degr, as shown in the bottom left corner. From this image, it is apparent that the extent of the background SMG is smaller than the beam size of the CARMA data, hence the CO($J$ = 3 $\rightarrow$ 2) detection is spatially unresolved.
Bottom: The Gaussian-fitted line peak of CO($J$ = 3 $\rightarrow$ 2) is 23.36 $\pm$ 2.21 mJy beam$^{-1}$ (before continuum
subtraction), and the integrated line flux over linewidth of $\sim$500 km/s is 14.6 $\pm$ 0.9 Jy km s$^{-1}$
(after continuum subtraction).  \label{fig:mom0}}
\end{figure*}

No significant continuum emission is detected from the line-free region down to a 1$\sigma$ limit of blah mJy. Neither CO(3-2) emission nor continuum emission are seen in the map near the NIR positions.

at $z$ = 2.685, 1$\arcsec \sim$ 8.1 kpc


\section{CO($J$ = 2 $\rightarrow$ 1) line emission}
\begin{figure*}[tbph]
\centering
\includegraphics[width=0.8\textwidth]{../OverlayFig/CO32mom0}
\includegraphics[width=0.65\textwidth]{../OverlayFig/2.685bin2_arcNS_further_Sblob}
\caption{The central cross represents the coordinates for alignment between plots. The contour starts at $\pm$2$\sigma$, incrementing at
steps of $\pm$1$\sigma$. Top Left: Continuum-subtracted moment-0 map of the CO($J$ = 3 $\rightarrow$ 2) line emission in the background SMG with $\sigma$ = 0.7 Jy km s$^{-1}$ beam$^{-1}$. The angular resolution is 13$\farcs$57 $\times$ 5\farcs77, at P.A. = -60.67$\degr$.
Top Right: Contours of CO($J$ = 3 $\rightarrow$ 2) emission (red) overlaid on the SMA 1 mm dust continuum (lime contours) with $\sigma$=0.836951 mJy beam$^{-1}$. The beam size of the SMA data is 1\farcs42$ \times $1\farcs16, P.A. -34.0\degr, as shown in the bottom left corner. From this image, it is apparent that the extent of the background SMG is smaller than the beam size of the CARMA data, hence the CO($J$ = 3 $\rightarrow$ 2) detection is spatially unresolved.
Bottom: The Gaussian-fitted line peak of CO($J$ = 3 $\rightarrow$ 2) is 23.36 $\pm$ 2.21 mJy beam$^{-1}$ (before continuum
subtraction), and the integrated line flux over linewidth of $\sim$500 km/s is 14.6 $\pm$ 0.9 Jy km s$^{-1}$
(after continuum subtraction).  \label{fig:mom0}}
\end{figure*}
\end{document}