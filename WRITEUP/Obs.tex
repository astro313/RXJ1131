
\documentclass[]{emulateapj}
\usepackage{amsmath}
\begin{document}
\author{Draft}


\section{Observations}
\subsection{PdBI} \label{sec:PdBIdata}
Observations of the CO($J$ = 2 $\rightarrow$ 1) rotational line ($\nu_{\rm
rest}$ = 230.5379938 GHz) toward the background galaxy RXJ1131-1231 at $z$ = 
0.655BLAH
 were carried out using IRAM Plateau de Bure Interferometer (PdBI) at the
redshifted frequency of $\nu_{\rm obs}$ = 
 BLAH GHz (BLAH mm) (Program ID: S14BX001; PI: D. Riechers). 
 Two observing runs were carried out on 2014 Dec 06 and 2015 Feb 05 under good
%2 mm
weather conditions in the C and D array configurations, respectively. The 2 mm
receivers were used to cover the redshifted CO($J$ = 2 $\rightarrow$ 1) line
and the 
underlying continuum emission (rest frame BLAH = (2.152 * (1+z)) mm), employing
a correlator setup providing an effective bandwidth of 3.16 GHz and spectral 
resolution of 10.0 MHz ($\sim$21.5281 km s$^{-1}$).
This resulted in 3.74 hours of cumulative six antenna-equivalent on-source time

after discarding unusable visibility data.
% D: 1.5 hours; 1055+018 , 3C279 (RF, fixed flux), 1127-145 (phase/amp),
1124-106(ditto)
% C: 2.2 hours; MWC349 (fixed flux), 1055+018 (RF = bandpass), 1124-186
(phase/amp), 1127-145 (ditto)
% total: 3.75 hours
%For both tracks, 
The nearby quasars 1127$-$145 and 1124$-$186 were observed every BLAH minutes
for
pointing, secondary amplitude, and phase calibration, and MWC349 was observed
as the primary
absolute flux calibrator for C array observations.
The nearby quasars 1127$-$145 and 1124$-$106 were observed every BLAH minutes
for
pointing, secondary amplitude, and phase calibration, and 3C279 was observed as
the primary
absolute flux calibrator for D array observations.
1055$+$018 and 3C279 were observed as bandpass calibrators for C and D array
observations, respectively, yielding $\lesssim
$15\% calibration accuracy.

The GILDAS package was used to reduce and analyze the visibility data which are
then imaged and deconvolved using
the CLEAN algorithm with ``natural" weighting. This yields a synthesized clean
beam size of 4$\farcs$44 $\times$ 1\farcs95. The final rms noise is $\sigma$ =
0.177 
Jy km s$^{-1}$ beam$^{-1}$ over a channel width of 320 MHz (corresponding to
690 km s$^{-1}$), and $\sigma$ = 1.451 Jy km s$^{-1}$ beam$^{-1}$ over 10 MHz 
(21.5 km s$^{-1}$) . 
The continuum image is created by % 2.152 mm; 
averaging over all the line-free channels ($\nu_{\rm cont}\sim$139 GHz). This
yields an rms noise of 0.082 mJy beam$^{-1}$. % see README.md in 04Sep15

\subsection{CARMA} \label{sec:carmadata}
% LO = 215.6694 Ghz
% 3C273 FG
% 3C279 Bandpass
% mars flux
% after flagging: Total observing time is  1.48 hours, checked with uvindex
%
Observations of the CO($J$ = 3 $\rightarrow$ 2) rotational line ($\nu_{\rm
rest}$ = 345.7959899 GHz) toward the background galaxy RXJ1131-1231 at $z$ = 
0.655BLAH 
 were carried out using the Combined Array for Research in Millimeter-wave
Astronomy (CARMA) at the redshifted frequency of $\nu_{\rm obs}$ = BLAH GHz
(BLAH 
mm) (Program ID: cf0098; PI: D. Riechers).
observations were carried out on 2014 February 17 under good 3 mm weather
conditions in the D array configurations. The 3 mm receivers were used to cover
the 
redshifted CO($J$ = 3 $\rightarrow$ 2) line, employing a correlator setup
providing a bandwidth of 3.75 GHz in each sideband and spectral resolution of
12.5 MHz ($
\sim$17BLAH km s$^{-1}$). The line was placed in the
lower sidebands with the local oscillator tuned to $\nu_{\rm LO}\sim$ 216 GHz;
this resulted in 1.5 hours of 15 antenna-equivalent on-source time after
discarding 
unusable visibility data. 

The nearby radio quasar 3C273 was observed every 15 minutes for
pointing, amplitude, and phase calibration, and Mars was observed as the
primary
absolute flux calibrator. 3C279 was observed as bandpass calibrators, yielding
$\sim
$15\%BLAH calibration accuracy.
The MIRIAD package was used to calibrate and analyze the visibility data which
are imaged and deconvolved using
the CLEAN algorithm with ``natural" weighting. This yields a synthesized clean
beam size of 3\farcs1 $\times$ 1\farcs9 for the lower sideband image cube. The
final 
rms noise is $\sigma$ = BLAH Jy km s$^{-1}$ beam$^{-1}$ over a channel width of
BKAH MHz (corresponding to BLAH km s$^{-1}$). 
The continuum image is created by
averaging over all the line-free channels ($\nu_{\rm cont}\sim$216 GHz). This
yields a synthesized clean beam size of Blah\farcs5 $\times$ blah\farcs0 and an
rms 
noise of 0.24Blah mJy beam$^{-1}$.

\subsection{VLA} 
Observations of the BLAH continuum ($\nu_{\rm rest}$ = BLAH GHz) toward the
galaxy RXJ1131-1231 were carried out using the {\it Karl G. Jansky} Very Large 
Array (VLA) at the redshifted frequency of $\nu_{\rm obs}$ = BLAH GHz (BLAH mm)
(Program ID: BLAH; PI: BLAH).
Observations were carried out on BLAH under excellent/GOOD? BLAH mm weather
conditions in the A array configurations. The C-Band receivers were used to 
cover the BLAH
, employing a correlator setup providing a bandwidth of BLAH GHz in each
sideband. This resulted in BLAH hours of BLAH antenna-equivalent on-source time
after 
discarding unusable visibility data.

The nearby radio quasar BLAH was observed every BLAH minutes for
pointing, amplitude, and phase calibration, and BLAH was observed as the
primary
absolute flux calibrator. BLAH were observed as bandpass calibrators, yielding
$\sim
$15\% calibration accuracy.
The AIPS package was used to calibrate and analyze the visibility data which
are imaged and deconvolved using
the CLEAN algorithm with ``natural" weighting. This yields a synthesized clean
beam size of 2$\farcs$6 $\times$ 2\farcs2 for the continuum image. The final
rms 
noise is $\sigma$ = 0.68 Jy km s$^{-1}$ beam$^{-1}$.

\end{document}