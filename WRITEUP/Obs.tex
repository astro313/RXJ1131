\documentclass[]{emulateapj}
\usepackage{amsmath}
\begin{document}
\author{Draft}


\section{Observations}
\subsection{PdBI} \label{sec:PdBIdata}
Observations of the CO($J$ = 2 $\rightarrow$ 1) rotational line ($\nu_{\rm rest}$ = 230.5379938 GHz) toward the background galaxy RXJ1131-1231 at $z$ = 0.655BLAH
 were carried out using IRAM Plateau de Bure Interferometer (PdBI) at the redshifted frequency of $\nu_{\rm obs}$ = 
 BLAH GHz (BLAH mm) (Program ID: S14BX001; PI: D. Riechers). 
 Two observing runs were carried out on 2014 Dec 06 and 2015 Feb 05 under excellent / GOOD? 
%2 mm
weather conditions in the C and D array configurations, respectively. The 2 mm receivers were used to cover the redshifted CO($J$ = 2 $\rightarrow$ 1) line and the underlying continuum emission (rest frame BLAH = (2.152 * (1+z)) mm), employing a correlator setup providing an effective bandwidth of 3.16 GHz and spectral resolution of 10.0 MHz ($\sim$21.5281 km s$^{-1}$).
This resulted in 3.74 hours of cumulative six antenna-equivalent on-source time 
after discarding unusable visibility data.
% D: 1.5 hours; 1055+018 , 3C279 (RF, fixed flux), 1127-145 (phase/amp), 1124-106(ditto)
% C: 2.2 hours; MWC349 (fixed flux), 1055+018 (RF = bandpass), 1124-186 (phase/amp), 1127-145 (ditto)
% total: 3.75 hours
%For both tracks, 
The nearby quasars 1127$-$145 and 1124$-$186 were observed every BLAH minutes for
pointing, secondary amplitude, and phase calibration, and MWC349 was observed as the primary
absolute flux calibrator for C array observations.
The nearby quasars 1127$-$145 and 1124$-$106 were observed every BLAH minutes for
pointing, secondary amplitude, and phase calibration, and 3C279 was observed as the primary
absolute flux calibrator for D array observations.
1055$+$018 and 3C279 were observed as bandpass calibrators for C and D array observations, respectively, yielding $\lesssim
$15\% calibration accuracy.

The GILDAS package was used to reduce and analyze the visibility data which are then imaged and deconvolved using
the CLEAN algorithm with ``natural" weighting. This yields a synthesized clean beam size of 4$\farcs$44 $\times$ 1\farcs95. The final rms noise is $\sigma$ = 0.177 Jy km s$^{-1}$ beam$^{-1}$ over a channel width of 320 MHz (corresponding to 690 km s$^{-1}$), and $\sigma$ = 1.451 Jy km s$^{-1}$ beam$^{-1}$ over 10 MHz (21.5 km s$^{-1}$) . 
The continuum image is created by % 2.152 mm; 
averaging over all the line-free channels ($\nu_{\rm cont}\sim$139 GHz). This yields an rms noise of 0.082 mJy beam$^{-1}$. % see README.md in 04Sep15

\subsection{CARMA} \label{sec:carmadata}
% LO = 215.6694 Ghz
% 3C273 FG
% 3C279 Bandpass
% mars flux
% after flagging: Total observing time is  1.48 hours, checked with uvindex
%
Observations of the CO($J$ = 3 $\rightarrow$ 2) rotational line ($\nu_{\rm rest}$ = 345.7959899 GHz) toward the background galaxy RXJ1131-1231 at $z$ = 
0.655BLAH 
 were carried out using the Combined Array for Research in Millimeter-wave Astronomy (CARMA) at the redshifted frequency of $\nu_{\rm obs}$ = BLAH GHz (BLAH mm) (Program ID: cf0098; PI: D. Riechers).
observations were carried out on 2014 February 17 under good 3 mm weather conditions in the D array configurations. The 3 mm receivers were used to cover the redshifted CO($J$ = 3 $\rightarrow$ 2) line, employing a correlator setup providing a bandwidth of 3.75 GHz in each sideband and spectral resolution of 12.5 MHz ($\sim$17BLAH km s$^{-1}$). The line was placed in the
lower sidebands with the local oscillator tuned to $\nu_{\rm LO}\sim$ 216 GHz; this resulted in 1.5 hours of 15 antenna-equivalent on-source time after discarding unusable visibility data. 

The nearby radio quasar 3C273 was observed every 15 minutes for
pointing, amplitude, and phase calibration, and Mars was observed as the primary
absolute flux calibrator. 3C279 was observed as bandpass calibrators, yielding $\sim
$15\%BLAH calibration accuracy.
The MIRIAD package was used to calibrate and analyze the visibility data which are imaged and deconvolved using
the CLEAN algorithm with ``natural" weighting. This yields a synthesized clean beam size of 3\farcs1 $\times$ 1\farcs9 for the lower sideband image cube. The final rms noise is $\sigma$ = BLAH Jy km s$^{-1}$ beam$^{-1}$ over a channel width of BKAH MHz (corresponding to BLAH km s$^{-1}$). 
The continuum image is created by
averaging over all the line-free channels ($\nu_{\rm cont}\sim$216 GHz). This yields a synthesized clean beam size of Blah\farcs5 $\times$ blah\farcs0 and an rms noise of 0.24Blah mJy beam$^{-1}$.

\subsection{VLA} 
Observations of the BLAH continuum ($\nu_{\rm rest}$ = BLAH GHz) toward the galaxy RXJ1131-1231 were carried out using the {\it Karl G. Jansky} Very Large Array (VLA) at the redshifted frequency of $\nu_{\rm obs}$ = BLAH GHz (BLAH mm) (Program ID: BLAH; PI: BLAH).
Observations were carried out on BLAH under excellent/GOOD? BLAH mm weather conditions in the A array configurations. The C-Band receivers were used to cover the BLAH
, employing a correlator setup providing a bandwidth of BLAH GHz in each sideband. This resulted in BLAH hours of BLAH antenna-equivalent on-source time after discarding unusable visibility data.

The nearby radio quasar BLAH was observed every BLAH minutes for
pointing, amplitude, and phase calibration, and BLAH was observed as the primary
absolute flux calibrator. BLAH were observed as bandpass calibrators, yielding $\sim
$15\% calibration accuracy.
The AIPS package was used to calibrate and analyze the visibility data which are imaged and deconvolved using
the CLEAN algorithm with ``natural" weighting. This yields a synthesized clean beam size of 2$\farcs$6 $\times$ 2\farcs2 for the continuum image. The final rms noise is $\sigma$ = 0.68 Jy km s$^{-1}$ beam$^{-1}$.
%
%\section{Continuum}
%%90.6797 GHz
%% VLA sigma =  Jy
%
%no continuum detection..
%Averaging over all the line-free channels, we did not detect the radio continuum down to 2$\sigma$ of blah mJy at an averaged frequency of $\nu_{cont}$ = 90.7 GHz.
%
%\begin{figure*}[tbph]
%\centering
%\includegraphics[width=0.80\textwidth]{../OverlayFig/ContPanel}
%\caption{Left: Contours of the CARMA 90.7 GHz continuum emission (red) in the foreground radio galaxy 3C220.3 overlaid on the image and contours of the VLA 6 GHz continuum emission (lime). The rms in the VLA 6 GHz continuum image is $\sigma$ = 0.0641 mJy beam$^{-1}$. The synthesized beam size is 0$\farcs$596 $\times$ 0$\farcs$229, at P.A. 75.79$\degr$. Lime contour levels start at $\pm$4$\times\sigma$ = x mJy beam$^{-1}$ and increment at steps of $\pm$2$\sigma$ for the VLA data; red contour levels start at $\pm$2$\sigma$, incrementing at steps of $\pm$1$\sigma$ = 0.24 mJy beam$^{-1}$.
%Right:
%Contour map of the 90.7 GHz continuum emission. The beam size is 16\farcs67$\pm
%$0\farcs69 $\times$ 7\farcs17$\pm$1\farcs60, at P.A. = -57.9$\degr$. The contour levels start at $\pm$2$\sigma$ and increment at steps of $\pm$1$\sigma$ =0.5 mJy beam$^{-1}$.\label{fig:cont}}
%\end{figure*}
%
%
%\section{New results: CO($J$ = 3 $\rightarrow$ 2) line emission}
%%- ! Parameters and Errors:
%%- ! MFIT%PAR[01]   10.289188385009766        1.7304190817956480
%%- ! MFIT%PAR[02]   51.276344299316406        29.649510734050978
%%- ! MFIT%PAR[03]   147.87234497070312 * 2.355        30.276954728539007
%%- ! MFIT%PAR[04]  0.34727144241333008       0.22560386117523606
%%- channel [, ]
%%- velocity range [,] km/s
%%- Integrated line flux =
%%- rms =
%
%%We have detected unresolved CO($J$ = 3 $\rightarrow$ 2) line emission toward SMM J0939+8315 at peak flux
%%of blah sigma  of $ \sigma$ = blah mJy beam$^{-1}$.
%We detect unresolved CO($J$ = 3 $\rightarrow$ 2) line emission toward the background SMG SMM J0939+8315. We extract the spectrum and fit a single-component Gaussian as shown in Figure \ref{fig:mom0}, yielding peak flux density of 10.3 $\pm$ 1.7 mJy, the FWHM of the Gaussian fit is 541.65 $\pm$ 31.91 km s$^{-1}$. The spatial extent of the SMG is shown in the SMA 1 mm dust continuum in Figure \ref{fig:mom0}, where the synthesized beam size of the SMA observation is 1\farcs42$\times$1\farcs16, at P.A. -34.0\degr. Since the angular resolution of our CO measurement is 13$\farcs$57 $\pm$ 1\farcs4 $\times$ 5\farcs77 $\pm$ 3\farcs1, the CO($J$ = 3 $\rightarrow$ 2) detection in the SMG is therefore spatially unresolved. We construct the velocity-integrated (moment-0) map of the CO($J$ = 3 $\rightarrow$ 2) emission using the uv-continuum subtracted data, the velocity-integrated CO($J$ = 3 $\rightarrow$ 2) line flux is 14.6 $\pm$ 0.9 Jy km s$^{-1}$.
%
%\begin{figure*}[tbph]
%\centering
%\includegraphics[width=0.8\textwidth]{../OverlayFig/CO32mom0}
%\includegraphics[width=0.65\textwidth]{../OverlayFig/2.685bin2_arcNS_further_Sblob}
%\caption{The central cross represents the coordinates for alignment between plots. The contour starts at $\pm$2$\sigma$, incrementing at
%steps of $\pm$1$\sigma$. Top Left: Continuum-subtracted moment-0 map of the CO($J$ = 3 $\rightarrow$ 2) line emission in the background SMG with $\sigma$ = 0.7 Jy km s$^{-1}$ beam$^{-1}$. The angular resolution is 13$\farcs$57 $\times$ 5\farcs77, at P.A. = -60.67$\degr$.
%Top Right: Contours of CO($J$ = 3 $\rightarrow$ 2) emission (red) overlaid on the SMA 1 mm dust continuum (lime contours) with $\sigma$=0.836951 mJy beam$^{-1}$. The beam size of the SMA data is 1\farcs42$ \times $1\farcs16, P.A. -34.0\degr, as shown in the bottom left corner. From this image, it is apparent that the extent of the background SMG is smaller than the beam size of the CARMA data, hence the CO($J$ = 3 $\rightarrow$ 2) detection is spatially unresolved.
%Bottom: The Gaussian-fitted line peak of CO($J$ = 3 $\rightarrow$ 2) is 23.36 $\pm$ 2.21 mJy beam$^{-1}$ (before continuum
%subtraction), and the integrated line flux over linewidth of $\sim$500 km/s is 14.6 $\pm$ 0.9 Jy km s$^{-1}$
%(after continuum subtraction).  \label{fig:mom0}}
%\end{figure*}
%
%No significant continuum emission is detected from the line-free region down to a 1$\sigma$ limit of blah mJy. Neither CO(3-2) emission nor continuum emission are seen in the map near the NIR positions.
%
%at $z$ = 2.685, 1$\arcsec \sim$ 8.1 kpc
%

\end{document}