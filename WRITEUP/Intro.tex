
\documentclass[]{emulateapj}
\usepackage{amsmath}

\usepackage[breaklinks,colorlinks,citecolor=blue,linkcolor=blue]{hyperref}
%urlcolor=
%
\usepackage{etoolbox}

\makeatletter

% Patch case where name and year are separated by aysep
\patchcmd{\NAT@citex}
  {\@citea\NAT@hyper@{%
     \NAT@nmfmt{\NAT@nm}%
     \hyper@natlinkbreak{\NAT@aysep\NAT@spacechar}{\@citeb\@extra@b@citeb}%
     \NAT@date}}
  {\@citea\NAT@nmfmt{\NAT@nm}%
   \NAT@aysep\NAT@spacechar\NAT@hyper@{\NAT@date}}{}{}

% Patch case where name and year are separated by opening bracket
\patchcmd{\NAT@citex}
  {\@citea\NAT@hyper@{%
     \NAT@nmfmt{\NAT@nm}%

\hyper@natlinkbreak{\NAT@spacechar\NAT@@open\if*#1*\else#1\NAT@spacechar\fi}%
       {\@citeb\@extra@b@citeb}%
     \NAT@date}}
  {\@citea\NAT@nmfmt{\NAT@nm}%

\NAT@spacechar\NAT@@open\if*#1*\else#1\NAT@spacechar\fi\NAT@hyper@{\NAT@date}}
  {}{}

\makeatother
%
\usepackage{xspace}
%
\usepackage{apjfonts}
\usepackage{amssymb, amsmath} % for e.g. \lesssim
\usepackage{natbibspacing, natbib}
\usepackage{aas_macros} % for understanding Journal in bib
\usepackage{graphics,graphicx}
%
\newcommand{\vdag}{(v)^\dagger}
\newcommand{\myemail}{tleung@astro.cornell.edu}
\newcommand{\Msun}{\mbox{M$_{\odot}$}\xspace}
\newcommand{\Rsun}{\mbox{R$_{\odot}$}\xspace}
\newcommand{\Lsun}{\mbox{L$_{\odot}$}\xspace}
\newcommand{\LIR}{\mbox{$L_{\rm IR}$}\xspace}
\newcommand{\LFIR}{\mbox{$L_{\rm FIR}$}\xspace}
%
\newcommand{\rarr}{$\rightarrow$}
\newcommand{\aco}{\mbox{CO($J$\,=\,1\,\rarr\,0) }}
\newcommand{\bco}{\mbox{CO($J$\,=\,2\,\rarr\,1) }}
\newcommand{\cco}{\mbox{CO($J$\,=\,3\,\rarr\,2) }}
\newcommand{\rot}[3][CO]{\mbox{#1($J$\,=\,#2\,\rarr\,#3)}}
%
\newcommand{\alphaco}{$\alpha_{\rm CO}$}
\newcommand{\cii}{[C{\scriptsize II}]}
\newcommand{\Lp}[1][CO]{\mbox{$L^{\prime}_\textrm{\fontsize{8pt}{12pt}\selectfont{#1}}$}}
\newcommand{\kms}{km\,s$^{-1}$\xspace}
\newcommand{\LpU}{\mbox{K\,\,km\,\,s$^{-1}$\,\,pc$^2$}\xspace}
\newcommand{\pmOne}{\mbox{$^{-1}$}\xspace}
% Numerical values
\newcommand{\E}[1]{$\times10^{#1}$}
\newcommand{\petm}[2]{$^{+#1}_{-#2}$}
%
\newcommand{\eg}{{e.g.,~}}
\newcommand{\ie}{{i.e.,~}}
%
\newcommand{\Fig}[1]{Figure~\ref{fig:#1}}
\newcommand{\Eq}[1]{Equation~\ref{eq:#1}}
\newcommand{\Tab}[1]{Table~\ref{tab:#1}}
\newcommand{\Sec}[1]{\S\ref{sec:#1}}
%
\newcommand\tna{\,\tablenotemark{a}}
\newcommand\tnb{\,\tablenotemark{b}}
\newcommand\tnc{\,\tablenotemark{c}}
\newcommand\tnd{\,\tablenotemark{d}}
\newcommand\tne{\,\tablenotemark{e}}
\newcommand\tnf{\,\tablenotemark{f}}
\newcommand\tng{\,\tablenotemark{g}}
\newcommand\tnh{\,\tablenotemark{h}}
\newcommand\tni{\,\tablenotemark{i}}
\newcommand\tnj{\,\tablenotemark{j}}
\newcommand\tnk{\,\tablenotemark{k}}
\newcommand\tnl{\,\tablenotemark{l}}
%
\def\herschel {{\it Herschel Space Observatory}\xspace}
\def\alma     {Atacama Large (sub-)Millimeter Array (ALMA)\xspace}
\def\spitzer {{\it Spitzer Space Observatory}\xspace}
\def\pdbi     {Plateau de Bure Interferometer\xspace}
\def\carma    {Combined Array for Research in Millimeter-wave Astronomy\xspace}
\def\cso      {Caltech Sumillimeter Observatory (CSO)\xspace}
\def\noema    {Northern Extended Millimeter Array (NOEMA)\xspace}
\def\vla      {{\it Karl G. Jansky} Very Large Array\xspace}
% Typography
\newcommand{\ncode}[1]{{\sc #1}}
% Codes / Softwares
\newcommand{\uvmcmcfit}{\ncode{uvmcmcfit}\xspace}
\def\aips {\ncode{AIPS}\xspace}
\def\casa {\ncode{CASA}\xspace}
% Long words
\newcommand{\lowZ}{low-metallicity\xspace}
\newcommand{\mulw}{multi-wavelength\xspace}
\newcommand{\SF}{star-formation\xspace}
\newcommand{\gl}{gravitationally lensed\xspace}
% Wavelength regimes
\newcommand{\fir}{far-IR\xspace}
\newcommand{\fuv}{far-UV\xspace}
\newcommand{\mir}{mid-IR\xspace}
\newcommand{\nir}{near-IR\xspace}
\slugcomment{To be submitted to the ApJ}

\citestyle{aa}
\shorttitle{TBD}
\shortauthors{Leung}

\begin{document}
%{\tiny  \RevisionInfo}

\title{}
\author{Draft}
\affil{Department of Astronomy, Space Sciences Building, Cornell University,
Ithaca, NY 14853, USA; \myemail}


\begin{abstract}
\end{abstract}

\section{Introduction} 
The gravitational lens RXJ 1131-1231 was discovered with spectroscopic
redshifts of the lens and the QSO are $z_{\rm L}$ = 0.295 and
$z_\textrm{AGN,HST}$ = 
0.658, respectively \citep{

% FG gal is an elliptical gal: Slues+03 absorption lines  


As detected in the HST images, RXJ1131-1231 is a quadruply imaged AGN with Einstein ring of size 1\farcs83 in radius. The 
background AGN is lensed into 4 point-like image A,B,C,D on a ring (emission from the AGN host galaxy) around G, the foreground galaxy. 
C06 report the redshift of the background AGN of 0.658 with magnification ~ 9, claimed Seyfert 1 spiral galaxy.




\section{Observations}
\subsection{PdBI} \label{sec:PdBIdata}
Observations of the CO($J$ = 2 $\rightarrow$ 1) rotational line ($\nu_{\rm
rest}$ = 230.5379938 GHz) toward the background galaxy RXJ1131-1231 at $z$ = 0.654
 were carried out using IRAM Plateau de Bure Interferometer (PdBI) at the
(Program ID: S14BX001; PI: D. Riechers). 
 Two observing runs were carried out on 2014 Dec 06 and 2015 Feb 05 under good
weather conditions in the C and D array configurations, respectively. The 2 mm
receivers were used to cover the redshifted CO($J$ = 2 $\rightarrow$ 1) line
and the 
underlying continuum emission (rest-frame 3.6 mm), employing
a correlator setup providing an effective bandwidth of 3.6 GHz and spectral 
resolution of 10.0 MHz ($\sim$21.5281 km s$^{-1}$).
This resulted in 3.74 hours of cumulative six antenna-equivalent on-source time
after discarding unusable visibility data.
% D: 1.5 hours; 1055+018 (BP) , 3C279 (RF, fixed flux), 1127-145 (phase/amp),
% 1124-186(ditto)
% C: 2.2 hours; MWC349 (fixed flux), 1055+018 (RF = bandpass), 1124-186
% (phase/amp), 1127-145 (ditto)
% total: 3.75 hours
%For both tracks, 
The nearby quasars 1127$-$145 and 1124$-$186 were observed every BLAH minutes
for
pointing, secondary amplitude, and phase calibration and 1055$+$018 was observed as bandpass calibrators 
for both tracks.
MWC349 and 3C279 were observed
as the primary
absolute flux calibrator for C and D array observations, respectively, yielding $\lesssim
$15\% calibration accuracy.

The GILDAS package was used to reduce and analyze the visibility data which are
then imaged and deconvolved using
the CLEAN algorithm with ``natural" weighting. This yields a synthesized clean
beam size of 4$\farcs$44 $\times$ 1\farcs95 (PA: 13\degr). The final rms noise is $\sigma$ =
0.177 
Jy km s$^{-1}$ beam$^{-1}$ over a channel width of 320 MHz (corresponding to
690 km s$^{-1}$), and $\sigma$ = 1.451 Jy km s$^{-1}$ beam$^{-1}$ over 10 MHz 
(21.5 km s$^{-1}$) . 
The continuum image is created by % 2.152 mm; 
averaging over all the line-free channels ($\nu_{\rm cont}\sim$139 GHz). This
yields an rms noise of 0.082 mJy beam$^{-1}$. % see README.md in 04Sep15

\subsection{CARMA} \label{sec:carmadata}
% LO = 215.6694 Ghz
% 1127-189 Gain for set 2
% 3C273 gain for set 3
% 3C279 Bandpass for both
% mars flux for both
% after flagging: Total observing time is  1.48 hours, checked with uvindex for
% set 3
% after flagging: Total obs. time is 2.94 hours, checked with uvindex for the
% combined set
% 12.500 GHz <=> 17.935 km/s
%
Observations of the CO($J$ = 3 $\rightarrow$ 2) rotational line ($\nu_{\rm
rest}$ = 345.7959899 GHz) toward the background galaxy RXJ1131-1231 at $z$ = 
0.654
 were carried out using the Combined Array for Research in Millimeter-wave
Astronomy (CARMA; Program ID: cf0098; PI: D. Riechers).
observations were carried out on 2014 February 17 under good 3 mm weather
conditions and on 2014 February 02 under terrible 3mm weather in the D array
configurations.  The 3 mm receivers were used to cover
the 
redshifted CO($J$ = 3 $\rightarrow$ 2) line, employing a correlator setup
providing a bandwidth of 3.75 GHz in each sideband and spectral resolution of
12.5 MHz ($
\sim$17.9 km s$^{-1}$). The line was placed in the
lower sidebands with the local oscillator tuned to $\nu_{\rm LO}\sim$ 216 GHz;
this resulted in 2.94 hours of 15 antenna-equivalent on-source time after
discarding 
unusable visibility data. 
The radio quasars 3C273 and J1127-189 was observed every 15 minutes for
pointing, amplitude, and phase calibration. Mars was observed as the
primary
absolute flux calibrator and 3C279 was observed as bandpass calibrators for
both tracks. 
The phase calibrator J1127-189 from the first track is faint and observed under
poor weather and the phase calibrator 3C273 is far from source for the second
track. Hence, the phase calibration was 
subpar, with rms scatter $\sim$ 60\degr, 
yielding
$\sim
$BLAH\% calibration accuracy.
% flux scale uncertainty
% gain variation over time
% phase scatter 
%
The MIRIAD package was used to calibrate and analyze the visibility data which
are imaged and deconvolved using
the CLEAN algorithm with ``natural" weighting. This yields a synthesized clean
beam size of 3\farcs2 $\times$ 1\farcs9 (PA: 8\degr) for the lower sideband
image cube. The
final 
rms noise is $\sigma$ = 13.3 mJy km s$^{-1}$ beam$^{-1}$ over a channel width
of
25 MHz (corresponding to 35.87 km s$^{-1}$). 

\subsection{VLA} 
% 20081229
% flux 1331+305
% phase 1130-149

Observations of the radio continuum toward the
galaxy RXJ1131-1231 were carried out using the {\it Karl G. Jansky} Very Large 
Array (VLA; Program ID: AW741; PI: Wucknitz).
Observations were carried out on BLAH under excellent/GOOD? weather
conditions in the A array configurations. The C-Band receivers were used to 
cover the BLAH
, employing a correlator setup providing a bandwidth of BLAH GHz in each
sideband. This resulted in BLAH hours of BLAH antenna-equivalent on-source time
after 
discarding unusable visibility data.

The nearby radio quasar BLAH was observed every BLAH minutes for
pointing, amplitude, and phase calibration, and BLAH was observed as the
primary
absolute flux calibrator. BLAH were observed as bandpass calibrators, yielding
$\sim
$15\% calibration accuracy.
The AIPS package was used to calibrate and analyze the visibility data which
are imaged and deconvolved using
the CLEAN algorithm with ``natural" weighting. This yields a synthesized clean
beam size of 2$\farcs$6 $\times$ 2\farcs2 for the continuum image. The final
rms 
noise is $\sigma$ = 0.68 Jy km s$^{-1}$ beam$^{-1}$.


\end{document}