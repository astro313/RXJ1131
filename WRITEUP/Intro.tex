Understanding the baryon cycle has immense implications to our understanding in galaxy evolution...
Despite the recent progress in theoretical models and observational studies...

%observed trend
Studies of galaxy evolution find SFRD and BHAH peak at
$z$$\sim$2 and declines rapidly toward $z$$=$0. A leading explanation for such
a trend is the evolution of molecular gas content and SFE in galaxies (CW13). Due to higher gas fraction than local, regardless of their position in the SFR-M* plane, both MS and SB at high z have depletion time that is much short than local, suggesting more efficient mode for SF. Recent studies also find dynamical evidence for clumpy, globally unstable disks at high-z,
where SF is enhanced by disk instabilities and not mergers (cite: Genzel+05, Forster-Schreiber+09).

While gas-rich systems also exist in local universe, they appear to be
fundamentally different from high-z galaxies.
The local gas-rich systems are also those with the highest IR luminosities, commonly known as ultra-luminous IR galaxies (ULIRGs; \LIR\,$>$\,10$^{12}$\Lsun), nearly of all which appear to be [interacting galaxies/galaxy mergers] in different stages of merging (e.g. Sanders & Mirabel 1996), powered by SB or AGN. While the nature of the power source are still not well-understood.
Regardless of the energy source ((merger-induced) starburst vs AGN accretion), both
are fueled by enormous concentration and supply of molecular gas that has been dynamically driven into the circumnuclear regions. However, it is still unclear what processes...


However, there is a dearth of CO measurements at intermediate redshifts due to observational difficulties, in which the \aco line is redshifted frequency range currently unavailable.
Studying the cold gas at the redshift range 0.2 < z < 1 is important though, as the most dramatic change in star formation activity (e.g. Madau et al. 1998, Hopkins & Beacom 2006). Up to now, very little is known about the molecular gas content at moderate z. The ULIRG sample of Solomon97 contains 37 objects, but only 2 at are z > 0.2. With the advance in mm-instruments, Combes+11 investigate the variation of molecuar gas and SFE in an expanded sample of 30 ULIRGs with CO observations.


% quenching
While we know blah at the local universe and studies of the gas properties at
high-z are also emerging in recent years (see CW13 for a review). It is still
unclear what mechanism is responsible for removing cold gas from galaxies and
shutting down their star formation to build-up present-day red sequence
galaxies, leading to a change of slope at high M* (esp. at $z$ $<$ 1.5),  i.e.
suppression of SFR at high M*, i.e. strong evolution in sSFR (of massive gal.)
(cite: Papovich, Whitaker+12, Magnelli+14, Schreiber+15, Lee+15).
% Many star formation suppression or "quenching" processes have been suggested, including mergers, ram pressure stripping, tidal ``harassment", etc. One of the more intriguing possibilities, however, is star formation quenching via powerful feedback from AGNs.


% formation of local gal. population
Physically motivated galaxy evolution models find that feedbacks are needed to turn off SF to build-up present day red sequence galaxies. (Dave+05)
Current studies agree that ULIRGs are important in giving rise to local
ellipticals (if mergerd) or spirals with massive bulge (Springel+05). But the formation and evolution is unclear due to the limited spatial resolution available in probing the gas of large sample at $z$$>$0. Ideally, to understand them would require high resolution to probe the kinematics and strucutral properties.
The role of merger in building up the mass of galaxies across the universe.

Based on theoretical grounds, the resulting morphology of the merger will depend mainly intial gas ontent and the mass ratio of the merging galaxies. Gas provides a dissipative component to remove angular momentum and assist colaspe/funnel to the center.s

 -- H2 gas more concentrated towards galactic nucleus in local ULIRGs, which are all mergers (e.g. Sanders & Mirabel 1996).
One explanation to the observed gas concentration toward center: gravitational torques as a result of interactions drive gas towards center towards the circumnuclear regions. SB and BH accretion can then be triggered if the dynamical time-scale is shorter than the feedback mechanisms.


% gas dynamics and AGN growth/ evolution
In the current paradigm, SMBHs grew in every massive galaxy during a luminous quasar phase, where distant QSOs are the
progenitors of the dormant SMBHs in nearby galaxies. Quasars are powered by gas falling into central BHs, when galaxies
collide or merge.
On the other hand, it is found that high-z BHs grow faster than their host galaxies since the dynamical mass inferred from CO velocity structure (line width) and spatial extent is lower than the inferred one based on local BH-bulge mass relation.
Meaning the Magorrian relation doesn't hold at $z$\,$\gtrsim$ BLAH, which means that we need to also understand the
properties of AGN host galaxies between these epochs.
From local observations, there is no strict relation between presence of an AGN, companions or barred structures, AGN don't
show obvious systemic signatures of nuclear fueling form host. Thus, need to trace dynamics of gas in QSO host galaxies to understand the dynamical processes leading to
gas accumulation/inflow toward center (e.g. bars or merger-driven) for SMBH grow and SB.


(ULIRGS; \LIR\,$>$\,10$^{12}$\Lsun) are typically and display morphological features/evidence such as double-nuclei (with close separation), tidal tails/features, distorted isophotes, rings/ring-like structure.


C06 report the with magnification ~ 9, claimed Seyfert 1 spiral galaxy.


%%%%%%%%%%%%%%%%%%%%%%%%%


\section{IRAC}
For the IRAC channels, the point spread function (PSF) is undersampled and thus photometry was extracted in circular apertures of radius 3?pixels (?3.6?arcsec) with background determined in an annulus of inner radius 12?pixels and outer radius 20?pixels (?14.4?24?arcsec). Apertures were centred on the location of each source as listed in Tables 1 and 2. Array location-dependent corrections were applied to the IRAC photometry. Correction images available online at http://ssc.spitzer.caltech.edu/irac/calibrationfiles/locationcolor/ were mosaicked in the same way as the data frames to produce a correction mosaic.

The aperture photometry of a source was then multiplied by the value of the correction mosaic at the centre of the source image. Aperture corrections were taken from tabulated values listed in the IRAC Data Handbook. Colour corrections were applied by interpolation from tabulated values using the source effective temperatures as listed in Tables 1 and 2.  Absolute calibration of the IRAC is stable to 1?3?per cent (Reach et al. 2005). We add a 3?per cent calibration uncertainty in quadrature to statistical background errors from pixel-to-pixel variation determined in the aperture module to give a final error on the IRAC photometry. The photometry was de-reddened using the values determined by Sanner et al. (2000) converted to IRAC wavelengths using relations of Flaherty et al. (2007). The final photometry is listed in Tables 3 and 4.

%%%%%%%%%%%%%%%%%%%%
CO/H2 conversion factor:
while alphaCO of 0.8 is advocated for ULIRGs.
The conversion factor in these dynamically turbulent system could be different than BLAH since the gas is potentially warmer, more excited, and denser; the conversion factor X\_CO scales as sqrt(n)/T.



